\documentclass{article}

\begin{document}

\title{Asymptotics}
\author{Spencer Hirsch, Tyler Gutowski, Remington Greko}
\date{\today}
\maketitle

\noindent You have been randomly assigned to teams. Work together to write a report
crossing this first bridge on algorithmic quest.

\medskip

\noindent Submit the team's report on Canvas. Include a task matric indicating who 
did what.

\begin{table}

\end{table}

\bigskip

\noindent \textbf{Asymptotic Quest}

\medskip

After successful completion of these exercises you will understand the topic of
\textit{Asymptotics} and be able to explain and correctly answer questions about
the topic.

\bigskip

\noindent  \textit{The Pieces and their relationships}

\medskip

The pieces are functions which we will call \textit{f, g,} and \textit{h,} should
we need others they can be named.

\medskip

\noindent Standard relations include:

\medskip

\begin{center}
    less than, equal, greater than, etc.
\end{center}

\noindent Relations can have properties such as:

\begin{center}
Reflexing, Symmetric, Transitive
\end{center}

\noindent Quantifiers are also needed

\begin{center}
For all, There exists...
\end{center}

\noindent Write precise (mathematical) definitionas of the following relations:

\begin{enumerate}
    \item Big-O:
    \item Big-$\Omega$:
    \item Big-$\Theta$:
\end{enumerate}

\noindent Give examples of functions that satisfy these relations.

\medskip

\noindent Explain how these relations describe bounds on running time (or other resources)
expended when an algorithm is executed on input of size \textit{n}.

\pagebreak

\begin{center}
    \begin{tabular}{|p{3cm}|p{6cm}|}
        \hline
        \textbf{Name} & \textbf{Section} \\
        \hline
        Remington Greko &  \\
        \hline
        Tyler Gutowski &  \\
        \hline
        Spencer Hirsch &  \\
        \hline
    \end{tabular}
\end{center}

\end{document}
