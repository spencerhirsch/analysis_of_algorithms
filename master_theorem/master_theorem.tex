\documentclass{article}
\usepackage{hyperref}
\usepackage[margin=1in]{geometry}
\usepackage{indentfirst}   % Indents first paragraph. change if u want ig
\usepackage{setspace}
\usepackage{qtree}
\doublespacing

\begin{document}
\title{\textbf{The Master Theorem}}
\author{Remington Greko, Tyler Gutowski, and Spencer Hirsch}
\date{\today}

\maketitle

\noindent Work with your team to write a report about the \textit{Master Theorem} for solving
a recurrence.

I've usually avoided the \textit{Master Theorem} because it has too many rules and does not
offer real insight to the solution of the recurrene. However, I do include it in my handouts
and the \textit{Master Therorem} may be useful for find the solution to \textit{Strassen's
matrix multiplication recurrence.}

\[T(n) = 7T(n/2) + O(n^2)\]

\noindent There are many details beneath this recurrecne. I strongly encourage you to read
what is in the Corman textbook and other sources.

Submit the team's report on Canvas. Include a task matric indicating who did what.

\pagebreak

\noindent \textbf{\href{https://en.wikipedia.org/wiki/Master_theorem_(analysis_of_algorithms)}{Wikipedia Summary of the Master Theorem}}

In 1980, the Master Theorem was proposed as the "unifying method" for solving recurrences by Jon Bentley, Dorothea Blostein, and James B. Saxe.\footnote{Wikipedia.}
Although the name may imply it can solve all recurrences, this is not the case, it's a generalized theory that may be helpful in
solving recurrence relations. The Master Theorem utilizes a divide and conquer approach with allows it to separate processing into
numerous parts. The master theorem can be expressed by adding the time taken by the top level process with the time made with the recursive
calls of the algorithm.\footnote{Id.} The algorithm for the recurrence relation is expressed as:\footnote{Id.}

\[T(n) + aT(n/b) + f(n)\]

\noindent Where, n is the input soze, a is the number of subproblems, and b is the factor by which the size is reduced.\footnote{Id.} It can be determined
based on the Theorem the best case time complexity of the Master Theorem is constant time or \textit{O(1)}.

\bigskip

\bigskip

\noindent \textbf{\href{https://brilliant.org/wiki/master-theorem/}{Brilliant Article on the Master Theorem}}





\pagebreak
\section{Works Cited}

``Master Theorem (Analysis of Algorithms).'' Wikipedia. Wikimedia Foundation, January 31, 2023. $https://en.wikipedia.org/wiki/Master_theorem_(analysis_of_algorithms)$.

``Master Theorem.'' Brilliant Math \&amp; Science Wiki. Accessed February 28, 2023. $https://brilliant.org/wiki/master-theorem/$. 

\pagebreak

\begin{center}
        \begin{tabular}{|p{3cm}|p{6cm}|}
            \hline
            \textbf{Name} & \textbf{Section} \\
            \hline
            Remington Greko & \\
            \hline
            Tyler Gutowski & \\
            \hline
            Spencer Hirsch & Read Wikipedia Article and Brilliant Article \\
            \hline
        \end{tabular}
    \end{center}

\end{document}