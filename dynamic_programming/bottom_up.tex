\documentclass{article}
\usepackage{hyperref}
\usepackage[margin=1in]{geometry}
\usepackage{indentfirst}   % Indents first paragraph. change if u want ig
\usepackage{setspace}
\doublespacing

\begin{document}
\title{\textbf{Dynamic Programming: Bottum-up Problem Solving}}
\author{Remington Greko, Tyler Gutowski, and Spencer Hirsch}
\date{\today}

\maketitle

\noindent \textbf{Bottom-up Problem Solving} 


Richard Bellman described this problem solving technique in the
1950’s and called it Dynamic Programming to impress his sponsors.
Dynamic Programming is an important problem solving paradigm.
Read about Bellman on Wikipedia and summarize the importance
of his contributions.

\medskip

Describe how dynamic programming works: Solve and memoriz-
ing solutions to small problems and use these solutions as building
blocks to construct a solution to a larger problem from the bottom-
up.

\bigskip

\noindent \textit{Example uses of Dynamic Programming}


Give at least three examples problems that use the dynamic program-
ming paradigm to solve problems. (You should be able to search
finding examples and summarize these in your group’s words.)

\bigskip

\noindent \textit{Good approximations to hard problems}

\medskip

There are Hard Problems: Problems that do not (seem) to have effi-
cient solutions. You will explore some of these later in future quests.

\medskip

There are dynamic programming algorithms that provide good ap-
proximate solutions to these hard problems. My memory tells me the
0–1 Knapsack Problem is an example of a hard problem with a good
dynamic programming approximation. Report on this example and
at least 2 additional hard problems with good approximate solutions.


\pagebreak+

\begin{center}
        \begin{tabular}{|p{3cm}|p{6cm}|}
            \hline
            \textbf{Name} & \textbf{Section} \\
            \hline
            Remington Greko &  \\
            \hline
            Tyler Gutowski &  \\
            \hline
            Spencer Hirsch & Example uses of Dynamic Programming \\
            \hline
        \end{tabular}
    \end{center}
    

\end{document}