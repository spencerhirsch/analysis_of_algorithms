\documentclass{article}
\usepackage{hyperref}
\usepackage[margin=1in]{geometry}
\usepackage{indentfirst}   % Indents first paragraph. change if u want ig
\usepackage{setspace}
\doublespacing

\begin{document}
\title{\textbf{Top-Down Problem Solving}}
\author{Remington Greko, Tyler Gutowski, and Spencer Hirsch}
\date{\today}

\maketitle

Work with your team to write a report showing your knowledge of
the Recursion.
Submit the team’s report on Canvas. Include a task matrix indicat-
ing who did what.

\textbf{Recursion}

There is a cute definition of recursion in the Hacker’s Dictionary: Recursion: See Recursion

\smallskip

There is a good description of recursion on Wikipedia, read it.
Top-down problem solving requires solving a recurrence relation.
There are a similarities between recurrence equations and ordinary
differential equations should you desire to explore.

\smallskip

After successful completion of this quest you will understand
how to model the time complexity of a recursive algorithm by a
recurrence of the form

\[T(n) = aT(n/b) + f(n)\]

together with some initial conditions to get things started.
Interpret the recurrence above as saying:
To solve a problem with input size n, solve a problem of size n/b
(you may need to do this a times) and apply a forcing function f(n)
at each step.

There are many ways to solve a recurrence:

\begin{enumerate}
    \item Guess or Given and Prove
    \item Unrolling also called substitution
    \item The Master Theorem
    \item Generating Functions
\end{enumerate}

\bigskip

\pagebreak

\begin{center}
        \begin{tabular}{|p{3cm}|p{6cm}|}
            \hline
            \textbf{Name} & \textbf{Section} \\
            \hline
            Remington Greko & Second example of Dynamic Programming $\&$ Shortest Path\\
            \hline
            Tyler Gutowski & Third Example of Dynamic Programming $\&$ Eight Queens Problem\\
            \hline
            Spencer Hirsch & How Dynamic Programming Works, One Example use of Dynamic Programming $\&$ 0-1 Knapsack Problem algorithm example \\
            \hline
        \end{tabular}
    \end{center}

\end{document}