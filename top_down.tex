\documentclass{article}
\usepackage{hyperref}
\usepackage[margin=1in]{geometry}
\usepackage{indentfirst}   % Indents first paragraph. change if u want ig
\usepackage{setspace}
\doublespacing

\begin{document}
\title{\textbf{Top-Down Problem Solving}}
\author{Remington Greko, Tyler Gutowski, and Spencer Hirsch}
\date{\today}

\maketitle

Work with your team to write a report showing your knowledge of
the Recursion.
Submit the team’s report on Canvas. Include a task matrix indicat-
ing who did what.

\textbf{Recursion}

There is a cute definition of recursion in the Hacker’s Dictionary: Recursion: See Recursion

\smallskip

There is a good description of recursion on Wikipedia, read it.
Top-down problem solving requires solving a recurrence relation.
There are a similarities between recurrence equations and ordinary
differential equations should you desire to explore.

\smallskip

After successful completion of this quest you will understand
how to model the time complexity of a recursive algorithm by a
recurrence of the form

\[T(n) = aT(n/b) + f(n)\]

together with some initial conditions to get things started.
Interpret the recurrence above as saying:
To solve a problem with input size n, solve a problem of size n/b
(you may need to do this a times) and apply a forcing function f(n)
at each step.

There are many ways to solve a recurrence:

\begin{enumerate}
    \item Guess or Given and Prove
    \item Unrolling also called substitution
    \item The Master Theorem
    \item Generating Functions
\end{enumerate}

\pagebreak

\noindent \textit{Guess or Given and Prove}

\smallskip

I like this approach to my third grade teacher Mrs. Beavis asking
me to prove x = 2 is a solution to the polynomial equation


\begin{enumerate}
    \item Show that log(n) (the log base 2 of n) solves the recurrence:
            \[T(n) = T(n/2) + 1, T(0) = 1\]
            Mention an algorithm that is described by this recurrence.
            (Note: you may assume n is a power of 2 so that dividing by 2
            never introduces a fraction)

            \textbf{Solution:}\\
            Assume n = $2^k$
            \[T(2^k) = T({2^k}/2) + 1\]
            \[T(2^k) = T(2^{k-1}) + 1\]
            \[T(2^k) = (T(2^{k-2}) + 1) + 1\]
            \[T(2^k) = T(2^{k-2}) + 2\]
            \[T(2^k) = [T(2^{k-3}) + 1] + 2\]
            \[T(2^k) = T(2^{k-3}) + 3\]
            \[T(2^k) = T(2^{k-k}) + k\]
            \[T(2^k) = T(2^{0}) + k\]
            \[T(2^k) = T(1) + k\]
            \[T(2^k) = 1 + k\]
            \[T(2^k) = log(n) + 1\]
            Therefore,\\
            \[T(n) = log(n)\]

            Binary search is a recursive alogorithm that has a time complexity
            of (log(n)).
        


    \item Show that nlog(n) solves the recurrence:
            \[T(n) = 2T(n/2) + n, T(0) = 1\]
            Mention a algorithm that is desribed by this recurrence.

            Mergesort is a recursive algorithm that has a time complexity 
            of O(nlog(n)).
\end{enumerate}

\bigskip

\pagebreak

\begin{center}
        \begin{tabular}{|p{3cm}|p{6cm}|}
            \hline
            \textbf{Name} & \textbf{Section} \\
            \hline
            Remington Greko & \\
            \hline
            Tyler Gutowski & \\
            \hline
            Spencer Hirsch & O(log(n)) Recurrence and Example Algorithms\\
            \hline
        \end{tabular}
    \end{center}

\end{document}