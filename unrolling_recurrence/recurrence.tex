\documentclass{article}
\usepackage{hyperref}
\usepackage[margin=1in]{geometry}
\usepackage{indentfirst}   % Indents first paragraph. change if u want ig
\usepackage{setspace}
\usepackage{qtree}
\doublespacing

\begin{document}
\title{\textbf{Unrolling a Recurrence}}
\author{Remington Greko, Tyler Gutowski, and Spencer Hirsch}
\date{\today}

\maketitle

\noindent Work with your team to write a report showing your knowledge of
the unrolling technique to solve a recurrence.

\smallskip

\noindent Submit the team's report of Canvas. Include a matrix indicating
who did what.

\noindent \textit{Sums}\\
\noindent Give a closed form expression for the sums below.

\begin{enumerate}
    \item 
          Sigma notation:
          \begin{equation}
            \sum_{i=0}^{i=n-1}1
          \end{equation}
          Closed form expression:
	  \begin{equation}
	    n
	  \end{equation}
    \item
          Sigma notation:
          \begin{equation}
            \sum_{i=0}^{i=n-1}2^i
          \end{equation}
          Closed form expression:
	  \begin{equation}
	    2-2^{n-1}
	  \end{equation}
    \item
          Sigma notation:
          \begin{equation}
            \sum_{k=0}^{k=n}(n!/k!(n-k)!)
          \end{equation}
          Closed form expression:
	  \begin{equation}
	    2^{n}
	  \end{equation}
    \item
          Sigma notation:
          \begin{equation}
            \sum_{k=0}^{k=n-1}r^k
          \end{equation}
          Closed form expression:
	  \begin{equation}
	   (r^{n}-1/(r-1)
	  \end{equation}
    \item
          Sigma notation:
          \begin{equation}
            \sum_{i=0}^{i=n-1}kr^{k-1}
          \end{equation}
          Closed form expression:
	  \begin{equation}
	  knr^{k-1}
	  \end{equation}
\end{enumerate}

\noindent \textbf{Unrolling a Recurrence}\\

\noindent I like this recurrence solving technique because it only involves
simple algebraic substitutions and an ability to compute closed form expressions
for common sums. Here is an example of unrolling that you can use as a template
for other problems.

Consider the recurrence:
\[ T(n) = T(n/2) + 1, T(0) = 1 \]
The idea is simple: if
\[ T(n) = T(n/2) + 1, T(1) = 0 \]
then by substitution (unrolling)
\[ T(n/2) = T(n/4) + 1, T(1) = 0 \]

\noindent n can be divded by 2 a only a finite number of times before 1 is
reached ending the unrolling. (It is olay to assume n = $2^p$ or p = logn and the
unrolling ends after p stages.) In another step
\[ T(n) = T(n/4) + 1 + 1, T(1) = 0 \]

\noindent Convince yourself this results in adding 1 to itself p times computing
the sume of values in the first column of Pascal's triangle.\\
Therefore,
\[ T(n) = p = log(n) \]

\noindent Make the connection that the recurrence models the binary search algorithm.\\
\noindent Here are some additional problems to flex and sharpen your skills.

\begin{enumerate}
    \item Solve the recurrence
            \[ T(n) = T(n-1) + n, T(0) = 1 \]
            \[ T(n) = T(n-1)^2 + n\]
            \[ T(n) = n^2 -2n + 1 + n\]
            \[ T(n) = n^2 - (2-1)n + 1\]
            \[ T(n) = n^2 - (2-1)n + 1\]
            \[T(n) = n^2\]

            Therefore, \[T(n) = n^2\]

            Mention the algorithm that is described by this recurrence.

            Bubble sort is an algorithm that has a recurrence of O($n^2$).
    \item Solve the recurrence
            \[ T(n) = 2T(n/2) + n, T(0) = 1\]
        
            \[ T(n) = 2(n/2)log(n/2) + n \]
            \[ T(n) = nlog(n/2) + n \]
            \[ T(n) = nlog((n) - nlog(2) + n) \]
            \[ T(n) = nlog(n) - n + n \]
            \[ T(n) = nlog(n) \]
            \[ T(n) = nlog(n) \]

            Therefore, \\
            \[T(n) = nlog(n)\]

            Mention the algorithm that is described by this recurrence.

            Mergesort is a recursive algoirthm that has a time complexity of O(nlog(n)).
    \item Solve the recurrence
            \[ T(n) = 2T(n-1) + 1, T(1) = 1 \]

            \[ T(n) = 2(2T(n-2) + 1) + 1 \]
            \[ T(n) = 2^2T(n-2) + 2 + 1 \]
            \[ T(n) = 2^3T(n-3) + 2^2 + 2 + 1 \]
            \[ T(n) = 2^kT(n-k) + 2^k-1 + 2^k-2 + ... + 2^2 + 2 + 1 \]
            \[ T(n) = 2^nT(0) + 1 + 2 + 2^2 + ... + 2^k-1 \]
            \[ T(n) = 2^n * 1 + 2^k - 1 \]
            \[ T(n) = 2^n + 2^n - 1 \]
            \[ T(n) = 2^n+1 - 1 \]
            
            Therefore, \\
            \[ T(n) = 2^n \]
            
            What famous problem is described by this recurrence?

            he Tower of Hanoi problem is a famous computer science problem which is 
            solved in an O($2^n$) time complexity.

\end{enumerate}


\pagebreak

\begin{center}
        \begin{tabular}{|p{3cm}|p{6cm}|}
            \hline
            \textbf{Name} & \textbf{Section} \\
            \hline
            Remington Greko & Unrolling recurrence part 2c\\
            \hline
            Tyler Gutowski & Closed form expression sums\\
            \hline
            Spencer Hirsch & Unrolling a recurrence part 2a and 2b \\
            \hline
        \end{tabular}
    \end{center}

\end{document}